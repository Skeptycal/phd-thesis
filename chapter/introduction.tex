\documentclass[a4paper,12pt,twoside]{../includes/ThesisStyle}
\usepackage[utf8]{inputenc}
\usepackage[T1]{fontenc}

\usepackage[left=1.5in,right=1.3in,top=1.1in,bottom=1.1in,includefoot,includehead,headheight=13.6pt]{geometry}\renewcommand{\baselinestretch}{1.05}


% =============================================================================
%\usepackage[sectionbib]{chapterbib}	% Cross-reference package (Natural BiB)
%\usepackage{bibunits}
%\usepackage{natbib}					% Put References at the end of each chapter
\usepackage{algorithm}
\usepackage{alltt}
\usepackage{amsfonts}
\usepackage{amsmath}
\usepackage{amssymb}
\usepackage{cite}
\usepackage{color}
\usepackage{enumerate}
\usepackage{booktabs} % used for \midrule
\usepackage{fancyhdr}					% Fancy Header and Footer
\usepackage{graphicx}
\usepackage{ifthen}
\usepackage{latexsym}
\usepackage{multirow}
\usepackage{rotating}					% Sideways of figures & tables
\usepackage{stmaryrd}
\usepackage{subfigure}
\usepackage{url}         
\usepackage{xspace}
\usepackage[normalem]{ulem} % for \sout
\usepackage{xcolor}
\usepackage{tablefootnote}
\usepackage{pifont}

% =============================================================================

% Table of contents for each chapter
\usepackage[nottoc, notlof, notlot]{tocbibind}
\usepackage{minitoc}
\setcounter{minitocdepth}{1}
\mtcindent=15pt

\setcounter{secnumdepth}{3}
\setcounter{tocdepth}{2}
  
% =============================================================================
% Fancy Header Style Options

\pagestyle{fancy}                       % Sets fancy header and footer
\fancyfoot{}                            % Delete current footer settings

%\renewcommand{\chaptermark}[1]{         % Lower Case Chapter marker style
%  \markboth{\chaptername\ \thechapter.\ #1}}{}} %

%\renewcommand{\sectionmark}[1]{         % Lower case Section marker style
%  \markright{\thesection.\ #1}}         %

\fancyhead[LE,RO]{\bfseries\thepage}    % Page number (boldface) in left on even
% pages and right on odd pages
\fancyhead[RE]{\bfseries\nouppercase{\leftmark}}      % Chapter in the right on even pages
\fancyhead[LO]{\bfseries\nouppercase{\rightmark}}     % Section in the left on odd pages

\let\headruleORIG\headrule
\renewcommand{\headrule}{\color{black} \headruleORIG}
\renewcommand{\headrulewidth}{1.0pt}
\usepackage{colortbl}
\arrayrulecolor{black}

\fancypagestyle{plain}{
  \fancyhead{}
  \fancyfoot{}
  \renewcommand{\headrulewidth}{0pt}
}


% =============================================================================
% Clear Header Style on the Last Empty Odd pages
\makeatletter

\def\cleardoublepage{\clearpage\if@twoside \ifodd\c@page\else%
  \hbox{}%
  \thispagestyle{empty}%              % Empty header styles
  \newpage%
  \if@twocolumn\hbox{}\newpage\fi\fi\fi}

\makeatother

\newenvironment{maxime}[1]
{
\vspace*{0cm}
\hfill
\begin{minipage}{0.5\textwidth}%
%\rule[0.5ex]{\textwidth}{0.1mm}\\%
\hrulefill $\:$ {\bf #1}\\
%\vspace*{-0.25cm}
\it 
}%
{%

\hrulefill
\vspace*{0.5cm}%
\end{minipage}
}

\let\minitocORIG\minitoc
\renewcommand{\minitoc}{\minitocORIG \vspace{1.5em}}


\renewcommand{\epsilon}{\varepsilon}

% centered page environment
\newenvironment{vcenterpage}
	{\newpage\vspace*{\fill}\thispagestyle{empty}\renewcommand{\headrulewidth}{0pt}}
	{\vspace*{\fill}}
	

%=============================================================================

\usepackage{needspace}
\newcommand{\needlines}[1]{\Needspace{#1\baselineskip}}

\usepackage{xcolor}
\definecolor{source}{gray}{0.95}
% source code formatting
\usepackage{listings}
    % global settings for source code listing package
\lstset{
    basicstyle=\ttfamily\small,
    showspaces=false,
    showstringspaces=false,
    captionpos=b, 
    columns=fullflexible}

\lstdefinelanguage{ST}{
    keywordsprefix=\#,
    morekeywords=[0]{true,false,nil},
    morekeywords=[1]{self,super,thisContext},
    morekeywords=[2]{ifTrue:,ifFalse:,whileTrue:,whileFalse:,and:,or:,xor:,not:,by:,timesRepeat:},
    sensitive=true,
    morecomment=[s]{"}{"},
    morestring=[d]',
    escapechar={!},
    alsoletter={., :, -, =, +, <},
    moredelim=**[is][\itshape]{/+}{+/},
    literate=
        {^}{{$\uparrow$}}1
        {:=}{{$\leftarrow$}}1
        {~}{{$\sim$}}1
        {-}{{\sf -\hspace{-0.13em}-}}1  % the goal is to make - the same width as +
        {+}{\raisebox{0.08ex}{+}}1		% and to raise + off the baseline to match V
        , % Don't forget the comma at the end!
    style=STStyle
}
\lstdefinestyle{STStyle}{
    tabsize=4,
    %frame=leftline,
    % frame=bl,
    %framerule=2pt,
    %rulecolor=\color{gray},
    % backgroundcolor=\color{white},
    %backgroundcolor=\usebeamercolor[bg]{listing},
    basicstyle=\ttfamily\small,
    keywordstyle=\bf\ttfamily,
    % stringstyle=\color{orange},
    stringstyle=\mdseries\slshape,
    commentstyle=\it\rmfamily\color{darkgray}, 
    commentstyle=\mdseries\slshape\color{gray},
    %commentstyle=\mdseries\slshape,
    emphstyle=\bf\ttfamily,
    escapeinside={!}{!},
	%backgroundcolor=\color{source},
    %emphstyle={[2]\color{red}},
    %emphstyle={[3]\color{blue}\bf},
    %emphstyle={[4]\color{blue}},
    keepspaces=true
} 

%\lstnewenvironment{javacode}  [1][]{\lstset{language=java,#1}\needlines{#2}}{} 
%\lstnewenvironment{pythoncode}[2][]{\lstset{language=python,#1}\needlines{#2}}{}
\lstnewenvironment{stcode}    [2][]{\lstset{language=ST,#1}\needlines{#2}}{}
\lstnewenvironment{ccode}     [2][]
    {\lstset{language=C,numbers=left,escapechar=\$,numberstyle=\tiny,#1}\needlines{#2}}{}

% ON: I tried to pass the line number options in as arg #1 but it does not work for me
% I also could net get the line numbers to consistently increase
\lstnewenvironment{numstcode} [2][]
    {\lstset{language=ST,numbers=left,numberstyle=\tiny,numbersep=2pt,#1}\needlines{#2}}{}
\lstnewenvironment{numstcodecont} [2][]
    {\lstset{language=ST,numbers=left,numberstyle=\tiny,numbersep=2pt,firstnumber=last#1}\needlines{#2}}{}

\newcommand{\lst}[1]{{\tt #1}}

% In-line code (literal)

% In-line code (latex enabled)
% Use this only in special situations where \ct does not work
% (within Section headings ...):
\newcommand{\lct}[1]{{\textsf{\textup{#1}}}}
% Code environments
\lstnewenvironment{code}{%
	\lstset{%
		% frame=lines,
		frame=single,
		framerule=0pt,
		mathescape=false
	}
}{}

%\renewcommand{\lstlistingname}{Code Example}

% =============================================================================
\newboolean{showcomments}
\setboolean{showcomments}{true}

\ifthenelse{\boolean{showcomments}} {
	\newcommand{\ugh}[1] {\textcolor{red}{\uwave{#1}}}	% please rephrase
	\newcommand{\ins}[1] {\textcolor{blue}{\uline{#1}}}	% please insert
	\newcommand{\del}[1] {\textcolor{red}{\sout{#1}}}	% please delete
	\newcommand{\chg}[2] {								% please change
		\textcolor{red}{\sout{#1}}{\ra}
		\textcolor{blue}{\uline{#2}}}
	\newcommand{\nbc}[3]{								% comment
		{\colorbox{#3}{\bfseries\sffamily\scriptsize\textcolor{white}{#1}}}
		{\textcolor{#3}{\sf\small$\blacktriangleright$\textit{#2}$\blacktriangleleft$}}}

}{
	\newcommand{\ugh}[1]{#1}							% please rephrase
	\newcommand{\ins}[1]{#1}							% please insert
	\newcommand{\del}[1]{}								% please delete
	\newcommand{\chg}[2]{#2}							% please change
	\newcommand{\nbc}[3]{}								% comment
}

% =============================================================================
\usepackage[pagebackref,hyperindex=true]{hyperref}


% Links in pdf
\usepackage{color}
\definecolor{linkcol}{rgb}{0.0, 0.0, 0.0} 
\definecolor{citecol}{rgb}{0.0, 0.0, 0.0} 

% Change this to change the informations included in the pdf file
% See hyperref documentation for information on those parameters
\hypersetup {
	bookmarksopen=true,
	pdftitle="Design and Use of Anatomical Atlases for Radiotherapy",
	pdfauthor="Olivier COMMOWICK", 
	pdfsubject="Creation of atlases and atlas based segmentation", %subject of the document
	%pdftoolbar=false, % toolbar hidden
	pdfmenubar=true, %menubar shown
	pdfhighlight=/O, %effect of clicking on a link
	colorlinks=true,
	pdfpagemode=UseNone,
	pdfpagelayout=SinglePage,
	pdffitwindow=true,
	linkcolor=linkcol,
	citecolor=citecol,
	urlcolor=linkcol
}

% =============================================================================
\newcommand{\figlabel}[1] {\label{fig:#1}}
\newcommand{\chaplabel}[1]{\label{chap:#1}}
\newcommand{\seclabel}[1] {\label{sec:#1}}
\newcommand{\tablabel}[1] {\label{tab:#1}}
\newcommand{\lstlabel}[1] {\label{lst:#1}}

\newcommand{\figref}[1] {Figure~\ref{fig:#1}}
\newcommand{\chapref}[1]{Chapter~\ref{sec:#1}}
\newcommand{\secref}[1] {Section~\ref{sec:#1}}
\newcommand{\tabref}[1] {Table~\ref{tab:#1}}
\newcommand{\lstref}[1] {Listing~\ref{tab:#1}}

\newcommand{\commented}[1]{}

\newcommand{\bs}    {\symbol{'134}} % backslash
\newcommand{\us}    {\symbol{'137}} % underscore
\newcommand{\ttt}[1]{\texttt{#1}}
\newcommand{\ie}    {\emph{i.e.},\xspace}
\newcommand{\eg}    {\emph{e.g.},\xspace}
\newcommand{\etal}  {\emph{et al.}\xspace}
\newcommand{\ns}    {\!\!\!\!} %big negative space
\newcommand{\cnull} {\textbackslash0\xspace}


\newcommand\fix[1]{\nb{FIX}{#1}}
\newcommand\todo[1]{\nb{TO DO}{#1}}
\newcommand\cb[1]{\nbc{CB}{#1}{purple}}
\newcommand\sd[1]{\nbc{SD}{#1}{orange}}
\newcommand\is[1]{\nbc{IS}{#1}{gray}}
\newcommand\gc[1]{\nbc{GC}{#1}{olive}}
\newcommand\ct[1]{\nbc{CT}{#1}{teal}}
\newcommand\md[1]{\nbc{MD}{#1}{blue}}
\newcommand\dc[1]{\nbc{DC}{#1}{green}}

% =============================================================================
\newcommand{\NBFFI}  {Native\-Boost-FFI\xspace}
\newcommand{\NB}  {Native\-Boost\xspace}
\newcommand{\B}   {Benzo\xspace}
\newcommand{\ST}  {Small\-talk\xspace}
\newcommand{\PH}  {Pharo\xspace}
\graphicspath{{.}{../figures/}}

\begin{document}
% =============================================================================
\chapter{Introduction}
\chaplabel{introduction}
\minitoc
% =============================================================================

\todo{Context}\\
\todo{outline existing work on unifying vms: klein, pinocchio}\\
\todo{- show contributions}\\
\todo{require complete modification of the language} \\
\todo{the basic execution scheme is not altered, only reified}\\
\todo{is it possible to do a bottom up approach?}\\
\todo{}

% =============================================================================
\section{Problems}
% =============================================================================
\todo{Sync with Chapter 2}
\begin{description}
	\item[Problem 1] Behavioral reflection comes at a significant cost
	\item[Problem 2] Intercession is limited to language-side
	\item[Problem 3] There is no unified reflection model between the VM and the language-side
\end{description}

% =============================================================================
\section{Contributions}
% =============================================================================
The contributions of this dissertation are:
\todo{maybe separate the abstract findings from the real software artifacts}
\begin{description}
	\item[\B] is a high-level low-level programming framework written in \urlfootnote{\PH}{http://pharo.org/}.
	The core functionality of \B is to dynamically execute native-code generated at language-side.
	Our framework requires minimal changes to an existing \VM and three custom primitives to support dynamic code activation, the majority of \B is implemented as accessible language-side code.
	\B allows us to directly communicate with the low-level world and thus hoist typical \VM-level applications to the language-side.
		
	\item[\NB] is a \B-based foreign function interface (\FFI).
	\NB makes generates customized native code at language-side, both being flexible and efficient at the same time.
	\NB outperforms other existing \FFI solutions on the \PH platform, making it an ideal evaluation for the \B framework.
	
	\item[\NBJ] is a prototype \JIT compiler based on \B.
	\NBJ generates the same native code as the \VM-level \JIT by compiling the high-level bytecode intermediate format at language-side.
	Our \B-based \JIT prototype reuses existing \VM-level infrastructure and focuses only on the dynamic code generation.
	However, since there is no well-defined interface with the \VM \NBJ requires an extended \VM with an improved \JIT interface to dynamically install native code.
	
	\item[\Eye] is a high-level inspector framework that has now been adopted in \PH.
	Inspectors are crucial when interacting with the low-level world which typically lacks the structural abstractions present at language-side.
	\Eye allows us to rapidly define customized views for \PH objects.
	Our inspector framework seamlessly integrates into the existing \PH debugger.
		
\end{description}


% =============================================================================
\section{Outline}
% =============================================================================

\begin{description}
\item[\chapref{background}] sheds light on the context of this work.
	We present a quick overview of language-side reflection followed by a development of \VM-level reflection.
	We find that mostly metacircular \VMs provide limited \VM-level reflection and thus we present several high-level language \VMs falling into this category.
	We conclude that there is only two research \VM that has a uniform model for \VM and language-side.
	Among them is \P a research \ST \VM we contributed to previous to working on this dissertation.

\item[\chapref{reification}] focuses on language-side applications that simplify the interaction with the underlying \VM.
	We present a custom inspector framework that is now used by default in \PH.
	As a second part we explain how we introduced first-class layouts and slots to \PH to reify the low-level structural layout of objects.
	Both projects are crucial for metacircular \VM development and are direct results from the research conducted on the \P \VM.

\item[\chapref{benzo}] describes a high-level low-level programming framework named \B.
	The core functionality of \B is to dynamically execute native-code generated at language-side.
	\B allows us to hoist typical \VM plugins to the language-side.
	Furthermore we show how code caching makes \B efficient and users essentially only pay a one-time overhead for generating the native code.
	
\item[\chapref{ffi}] presents \NB, a stable foreign function interface (\FFI) implementation that is entirely written at language-side using \B.
	\NB is a real-world validation of \B as it combines both language-side flexibility with \VM-level performance.
	We show in detail how \NB outperforms other existing \FFI solutions on \PH.

\item[\chapref{validation}] focuses on two further \B applications.
	In the first part we present \WF a framework for dynamically generating primitives at runtime.
	\WF extends the concept of metacircularity to the running language by reusing the same sources for dynamic primitives that were previously used to generate the static \VM artifact.
	In a first validation we show how \WF outperforms other reflective language-side solutions to instrument primitives.
	
	In a second part of \chapref{validation} we present \NBJ a prototype \JIT compiler that is based on \B.
	\NBJ shows the limitations of the \B approach as it required a customized \VM to communicate with the existing \JIT interface for native code.
	Our prototype implementation generates the same native code as the existing \VM-level \JIT, however, it is currently limited to simple expressions.
	\NBJ shows that for certain applications a well-define interface with the low-level components of the \VM is required.

\item[\chapref{future}] summarizes the limitations of \B and its application, furthermore we list undergoing efforts on the \B infrastructure and future work.

\item[\chapref{conclusion}] concludes the dissertation.

\end{description}

% =============================================================================
\ifx\wholebook\relax\else
    \end{document}
\fi
% =============================================================================