\documentclass[a4paper,12pt,twoside]{../includes/ThesisStyle}
\usepackage[utf8]{inputenc}
\usepackage[T1]{fontenc}

\usepackage[left=1.5in,right=1.3in,top=1.1in,bottom=1.1in,includefoot,includehead,headheight=13.6pt]{geometry}\renewcommand{\baselinestretch}{1.05}


% =============================================================================
%\usepackage[sectionbib]{chapterbib}	% Cross-reference package (Natural BiB)
%\usepackage{bibunits}
%\usepackage{natbib}					% Put References at the end of each chapter
\usepackage{algorithm}
\usepackage{alltt}
\usepackage{amsfonts}
\usepackage{amsmath}
\usepackage{amssymb}
\usepackage{cite}
\usepackage{color}
\usepackage{enumerate}
\usepackage{booktabs} % used for \midrule
\usepackage{fancyhdr}					% Fancy Header and Footer
\usepackage{graphicx}
\usepackage{ifthen}
\usepackage{latexsym}
\usepackage{multirow}
\usepackage{rotating}					% Sideways of figures & tables
\usepackage{stmaryrd}
\usepackage{subfigure}
\usepackage{url}         
\usepackage{xspace}
\usepackage[normalem]{ulem} % for \sout
\usepackage{xcolor}
\usepackage{tablefootnote}
\usepackage{pifont}

% =============================================================================

% Table of contents for each chapter
\usepackage[nottoc, notlof, notlot]{tocbibind}
\usepackage{minitoc}
\setcounter{minitocdepth}{1}
\mtcindent=15pt

\setcounter{secnumdepth}{3}
\setcounter{tocdepth}{2}
  
% =============================================================================
% Fancy Header Style Options

\pagestyle{fancy}                       % Sets fancy header and footer
\fancyfoot{}                            % Delete current footer settings

%\renewcommand{\chaptermark}[1]{         % Lower Case Chapter marker style
%  \markboth{\chaptername\ \thechapter.\ #1}}{}} %

%\renewcommand{\sectionmark}[1]{         % Lower case Section marker style
%  \markright{\thesection.\ #1}}         %

\fancyhead[LE,RO]{\bfseries\thepage}    % Page number (boldface) in left on even
% pages and right on odd pages
\fancyhead[RE]{\bfseries\nouppercase{\leftmark}}      % Chapter in the right on even pages
\fancyhead[LO]{\bfseries\nouppercase{\rightmark}}     % Section in the left on odd pages

\let\headruleORIG\headrule
\renewcommand{\headrule}{\color{black} \headruleORIG}
\renewcommand{\headrulewidth}{1.0pt}
\usepackage{colortbl}
\arrayrulecolor{black}

\fancypagestyle{plain}{
  \fancyhead{}
  \fancyfoot{}
  \renewcommand{\headrulewidth}{0pt}
}


% =============================================================================
% Clear Header Style on the Last Empty Odd pages
\makeatletter

\def\cleardoublepage{\clearpage\if@twoside \ifodd\c@page\else%
  \hbox{}%
  \thispagestyle{empty}%              % Empty header styles
  \newpage%
  \if@twocolumn\hbox{}\newpage\fi\fi\fi}

\makeatother

\newenvironment{maxime}[1]
{
\vspace*{0cm}
\hfill
\begin{minipage}{0.5\textwidth}%
%\rule[0.5ex]{\textwidth}{0.1mm}\\%
\hrulefill $\:$ {\bf #1}\\
%\vspace*{-0.25cm}
\it 
}%
{%

\hrulefill
\vspace*{0.5cm}%
\end{minipage}
}

\let\minitocORIG\minitoc
\renewcommand{\minitoc}{\minitocORIG \vspace{1.5em}}


\renewcommand{\epsilon}{\varepsilon}

% centered page environment
\newenvironment{vcenterpage}
	{\newpage\vspace*{\fill}\thispagestyle{empty}\renewcommand{\headrulewidth}{0pt}}
	{\vspace*{\fill}}
	

%=============================================================================

\usepackage{needspace}
\newcommand{\needlines}[1]{\Needspace{#1\baselineskip}}

\usepackage{xcolor}
\definecolor{source}{gray}{0.95}
% source code formatting
\usepackage{listings}
    % global settings for source code listing package
\lstset{
    basicstyle=\ttfamily\small,
    showspaces=false,
    showstringspaces=false,
    captionpos=b, 
    columns=fullflexible}

\lstdefinelanguage{ST}{
    keywordsprefix=\#,
    morekeywords=[0]{true,false,nil},
    morekeywords=[1]{self,super,thisContext},
    morekeywords=[2]{ifTrue:,ifFalse:,whileTrue:,whileFalse:,and:,or:,xor:,not:,by:,timesRepeat:},
    sensitive=true,
    morecomment=[s]{"}{"},
    morestring=[d]',
    escapechar={!},
    alsoletter={., :, -, =, +, <},
    moredelim=**[is][\itshape]{/+}{+/},
    literate=
        {^}{{$\uparrow$}}1
        {:=}{{$\leftarrow$}}1
        {~}{{$\sim$}}1
        {-}{{\sf -\hspace{-0.13em}-}}1  % the goal is to make - the same width as +
        {+}{\raisebox{0.08ex}{+}}1		% and to raise + off the baseline to match V
        , % Don't forget the comma at the end!
    style=STStyle
}
\lstdefinestyle{STStyle}{
    tabsize=4,
    %frame=leftline,
    % frame=bl,
    %framerule=2pt,
    %rulecolor=\color{gray},
    % backgroundcolor=\color{white},
    %backgroundcolor=\usebeamercolor[bg]{listing},
    basicstyle=\ttfamily\small,
    keywordstyle=\bf\ttfamily,
    % stringstyle=\color{orange},
    stringstyle=\mdseries\slshape,
    commentstyle=\it\rmfamily\color{darkgray}, 
    commentstyle=\mdseries\slshape\color{gray},
    %commentstyle=\mdseries\slshape,
    emphstyle=\bf\ttfamily,
    escapeinside={!}{!},
	%backgroundcolor=\color{source},
    %emphstyle={[2]\color{red}},
    %emphstyle={[3]\color{blue}\bf},
    %emphstyle={[4]\color{blue}},
    keepspaces=true
} 

%\lstnewenvironment{javacode}  [1][]{\lstset{language=java,#1}\needlines{#2}}{} 
%\lstnewenvironment{pythoncode}[2][]{\lstset{language=python,#1}\needlines{#2}}{}
\lstnewenvironment{stcode}    [2][]{\lstset{language=ST,#1}\needlines{#2}}{}
\lstnewenvironment{ccode}     [2][]
    {\lstset{language=C,numbers=left,escapechar=\$,numberstyle=\tiny,#1}\needlines{#2}}{}

% ON: I tried to pass the line number options in as arg #1 but it does not work for me
% I also could net get the line numbers to consistently increase
\lstnewenvironment{numstcode} [2][]
    {\lstset{language=ST,numbers=left,numberstyle=\tiny,numbersep=2pt,#1}\needlines{#2}}{}
\lstnewenvironment{numstcodecont} [2][]
    {\lstset{language=ST,numbers=left,numberstyle=\tiny,numbersep=2pt,firstnumber=last#1}\needlines{#2}}{}

\newcommand{\lst}[1]{{\tt #1}}

% In-line code (literal)

% In-line code (latex enabled)
% Use this only in special situations where \ct does not work
% (within Section headings ...):
\newcommand{\lct}[1]{{\textsf{\textup{#1}}}}
% Code environments
\lstnewenvironment{code}{%
	\lstset{%
		% frame=lines,
		frame=single,
		framerule=0pt,
		mathescape=false
	}
}{}

%\renewcommand{\lstlistingname}{Code Example}

% =============================================================================
\newboolean{showcomments}
\setboolean{showcomments}{true}

\ifthenelse{\boolean{showcomments}} {
	\newcommand{\ugh}[1] {\textcolor{red}{\uwave{#1}}}	% please rephrase
	\newcommand{\ins}[1] {\textcolor{blue}{\uline{#1}}}	% please insert
	\newcommand{\del}[1] {\textcolor{red}{\sout{#1}}}	% please delete
	\newcommand{\chg}[2] {								% please change
		\textcolor{red}{\sout{#1}}{\ra}
		\textcolor{blue}{\uline{#2}}}
	\newcommand{\nbc}[3]{								% comment
		{\colorbox{#3}{\bfseries\sffamily\scriptsize\textcolor{white}{#1}}}
		{\textcolor{#3}{\sf\small$\blacktriangleright$\textit{#2}$\blacktriangleleft$}}}

}{
	\newcommand{\ugh}[1]{#1}							% please rephrase
	\newcommand{\ins}[1]{#1}							% please insert
	\newcommand{\del}[1]{}								% please delete
	\newcommand{\chg}[2]{#2}							% please change
	\newcommand{\nbc}[3]{}								% comment
}

% =============================================================================
\usepackage[pagebackref,hyperindex=true]{hyperref}


% Links in pdf
\usepackage{color}
\definecolor{linkcol}{rgb}{0.0, 0.0, 0.0} 
\definecolor{citecol}{rgb}{0.0, 0.0, 0.0} 

% Change this to change the informations included in the pdf file
% See hyperref documentation for information on those parameters
\hypersetup {
	bookmarksopen=true,
	pdftitle="Design and Use of Anatomical Atlases for Radiotherapy",
	pdfauthor="Olivier COMMOWICK", 
	pdfsubject="Creation of atlases and atlas based segmentation", %subject of the document
	%pdftoolbar=false, % toolbar hidden
	pdfmenubar=true, %menubar shown
	pdfhighlight=/O, %effect of clicking on a link
	colorlinks=true,
	pdfpagemode=UseNone,
	pdfpagelayout=SinglePage,
	pdffitwindow=true,
	linkcolor=linkcol,
	citecolor=citecol,
	urlcolor=linkcol
}

% =============================================================================
\newcommand{\figlabel}[1] {\label{fig:#1}}
\newcommand{\chaplabel}[1]{\label{chap:#1}}
\newcommand{\seclabel}[1] {\label{sec:#1}}
\newcommand{\tablabel}[1] {\label{tab:#1}}
\newcommand{\lstlabel}[1] {\label{lst:#1}}

\newcommand{\figref}[1] {Figure~\ref{fig:#1}}
\newcommand{\chapref}[1]{Chapter~\ref{sec:#1}}
\newcommand{\secref}[1] {Section~\ref{sec:#1}}
\newcommand{\tabref}[1] {Table~\ref{tab:#1}}
\newcommand{\lstref}[1] {Listing~\ref{tab:#1}}

\newcommand{\commented}[1]{}

\newcommand{\bs}    {\symbol{'134}} % backslash
\newcommand{\us}    {\symbol{'137}} % underscore
\newcommand{\ttt}[1]{\texttt{#1}}
\newcommand{\ie}    {\emph{i.e.},\xspace}
\newcommand{\eg}    {\emph{e.g.},\xspace}
\newcommand{\etal}  {\emph{et al.}\xspace}
\newcommand{\ns}    {\!\!\!\!} %big negative space
\newcommand{\cnull} {\textbackslash0\xspace}


\newcommand\fix[1]{\nb{FIX}{#1}}
\newcommand\todo[1]{\nb{TO DO}{#1}}
\newcommand\cb[1]{\nbc{CB}{#1}{purple}}
\newcommand\sd[1]{\nbc{SD}{#1}{orange}}
\newcommand\is[1]{\nbc{IS}{#1}{gray}}
\newcommand\gc[1]{\nbc{GC}{#1}{olive}}
\newcommand\ct[1]{\nbc{CT}{#1}{teal}}
\newcommand\md[1]{\nbc{MD}{#1}{blue}}
\newcommand\dc[1]{\nbc{DC}{#1}{green}}

% =============================================================================
\newcommand{\NBFFI}  {Native\-Boost-FFI\xspace}
\newcommand{\NB}  {Native\-Boost\xspace}
\newcommand{\B}   {Benzo\xspace}
\newcommand{\ST}  {Small\-talk\xspace}
\newcommand{\PH}  {Pharo\xspace}
\graphicspath{{.}{../figures/}}

\begin{document}
% =============================================================================
\chapter{Introduction}
\chaplabel{introduction}
\minitoc
% =============================================================================

\todo{Recheck intro}
\noindent Reflection as common feature has been adopted by several dynamic high-level languages such as \urlfootnote{\Ruby}{http://rubylang.org/}, \urlfootnote{\Python}{http://python.org/} or \urlfootnote{\PH}{http://pharo.org/}.
Even some less dynamic languages such as \Java support a broad range of reflective features.
In the context of this dissertation we will mainly refer to the \PH environment which inherited its reflective capability from \ST.
One major reservation of programming languages over reflection is performance.
Reflection requires support from the underlying \VM and introduces another level of late-binding which prevent more aggressive static optimizations.

%\todo{probably drop this paragraph}
%Typically we distinguish two types of reflection: structural and behavioral \cite{Maes87a}.
%Structural reflection is concerned with the static structure of a program, while behavioral reflection focuses on the dynamic part of a running program.
%Orthogonally to the previous categorization we distinguish between introspection and intercession. 
%For introspection we only access a reified concept, whereas for intercession we alter the reified representation.

Especially behavioral reflection is costly, as it does not access the statically defined structure but requires dynamically reified representations of the execution context of a program.
There several approaches to limit the costs of reflection and the impact on the whole language runtime:
\begin{enumerate}[nolistsep]
	\item Limit the amount of reification performed,
	\item Restrict the reflective capabilities,
	\item Optimize the \VM to lower the reification overhead.
\end{enumerate}
The latter case allows more optimizations and is typically the path chose for more performance oriented programming languages.
For \PH only the first option is feasible as the integrated development environment makes ready use of reflection.
The first approach results in \emph{partial reflection} or \emph{partial behavior reflection} \cite{Tant03a}.
Typically the code locations where reification is necessary do not change often and can thus by dynamically optimized at runtime \cite{Roet07b}.

The research on partial behavioral reflection shows the real world limitations of reflective applications.
For many applications the reification overhead is too big.
The third approach to limit the costs is by providing specific support for a required reification.
An example for that is the debugging capability of many programming languages.
Most programming have special \VM support for this.
In contrast, \PH or \ST systems the debugger is built on top of the default reflection infrastructure.
\todo{are there other similar possiblities}

In general we require \VM-level support for efficient reflection.
Intercession is contained at language-side and there is usually no interface present to reach the \VM.
Hence, custom support is only possible by statically modifying the underlying \VM upfront.
Even in a theoretical setup where the \VM is open for language-side modification we see that hardly any \VM is self-aware.
Most \VMs behave at runtime like any other binary and lack most structural information required for reflection.
A partial exception to this are \VMs written on top of \VM frameworks that use high-level languages to describe the \VM components instead of the prevailing static system programming languages such as C or C++.
Even though, such \VM frameworks allow for compile-time reflection, the reflective capabilities are lost during compilation.

All these \VMs have in common that they strongly isolate the language running on top.
Recently several language-runtimes appeared that no longer make this strong separation \cite{Unga05a, Verw12a}.
One such system is the \P \ST \VM.
Instead of using bytecodes as language-side execution format \P's compiler directly generates native code for execution.
Essentially \P hoist the \VM-level just-in-time compiler (\JIT) to the language-side.
At the same time the underlying \VM is reduced to a bootstrap routine that hands over execution to the \P generated native code.
The direct usage of native code allows us to specify the lookup routine in plain \ST code at language-side.

Another project that follows the same principles as \P is the \Klein \VM.
Both language-runtimes have a reified base-level, the \VM components are represented as language-side structures.
This allows us to modify and inspect the \VM, it is reflective.
Both \Klein and \P are whole-system bottom up approaches that add powerful hooks at language-side.
Is it possible to achieve similar functionality with a simpler interface?
What are the limiting factors for such an approach?

To answer these questions we propose the high-level low-level programming framework \B written for \PH.
In the core \B allows us to dynamically activate native code from language-side.
This adds a primitive yet generic interface to the low-level \VM world.
To validate \B we describe three distinct applications built on top of it.
%
\begin{description}
	\item[\FFI:] The first one is an efficient foreign function interface (\FFI) library that is built at language-side without additional \VM support.
	Our \FFI library outperforms existing solutions on \PH.
	
	\item[Dynamic Primitives:] The second applications uses \B to dynamically generate and modify \PH primitives by reusing the metacircular \VM sources.
	By combining high-level reflection and \B's low-level performance we outperform pure \PH-based primitive instrumentation.
	
	\item[Language-side \JIT:] As a third, prototype application we show how \B is used to build a language-side \JIT.
	Our prototype shows the limits of possible \VM interactions using the \B framework. 
\end{description}


% =============================================================================
\section{Problem Statement}
% =============================================================================
Following the problem description listed in the above introduction we identified the following abstract concerns with existing reflective languages and their \VMs.
%
\todo{two step lowering: 1st theoretical problems, 2nd problems we solve}
\begin{enumerate}
	\item Reflection and in special behavioral reflection comes at a significant cost due to reification overhead.
		
	\item Intercession is limited to language-side.
	\VMs are not accessible from language-side and they are usually have no reflective properties at runtime. 
	
	\item Existing approaches to a unified model between the VM and the language-side are holistic, there is no intermediate solution available.
\end{enumerate}

\noindent Out of these general problems concerning reflection in high-level languages we see that they have a low-level root.
To address the unification of the language-side and \VM-side we have to grant more access to the language-side.
This includes interacting directly with low-level native instructions.
A similar problem has been solved by applying high-level low-level programming in a more static environment \cite{Fram09a}.
The approach outlined by Frampton et al. uses a high-level framework to generate native code at compile-time.
We see that their approach has not yet been applied in a more dynamic environment where native code has to be generated at runtime.
Hence we focus on the following concrete problems we wish to solve in this thesis.

\begin{description}
	\item[Problem 1] High-level low-level programming is not available at runtime.
	
	\item[Problem 2] Intercession is limited to language-side.
	\VMs are not accessible from language-side and they are usually have no reflective properties at runtime.
	
	\item[Problem 3] High-level low-level programming has not yet been applied to implement \VM-level components at language-side.
\end{description}


% =============================================================================
\section{Contributions}
% =============================================================================

We present now our contributions of this dissertation addressing the previously identified problems concerning high-level low-level programming in a dynamic language:

\todo{maybe separate the abstract findings from the real software artifacts}
\begin{description}
	\item[\B] is a high-level low-level programming framework written in \urlfootnote{\PH}{http://pharo.org/}.
	The core functionality of \B is to dynamically execute native-code generated at language-side.
	Our framework requires minimal changes to an existing \VM and three custom primitives to support dynamic code activation, the majority of \B is implemented as accessible language-side code.
	\B allows us to directly communicate with the low-level world and thus hoist typical \VM-level applications to the language-side.
		
	\item[\NB] is a \B-based foreign function interface (\FFI).
	\NB makes generates customized native code at language-side, both being flexible and efficient at the same time.
	\NB outperforms other existing \FFI solutions on the \PH platform, making it an ideal evaluation for the \B framework.
	
	\item[\NBJ] is a prototype \JIT compiler based on \B.
	\NBJ generates the same native code as the \VM-level \JIT by compiling the high-level bytecode intermediate format at language-side.
	Our \B-based \JIT prototype reuses existing \VM-level infrastructure and focuses only on the dynamic code generation.
	However, since there is no well-defined interface with the \VM \NBJ requires an extended \VM with an improved \JIT interface to dynamically install native code.
	
	\item[\AsmJIT] is a assembler framework written in \PH.
	\AsmJIT is the low-level backend for the previously mentioned \B framework.
	We extended the existing assembler framework to support the full 64-bit x86 instruction set.
	
	\item[\Eye] is a high-level inspector framework that has now been adopted in \PH.
	Inspectors are crucial when interacting with the low-level world which typically lacks the structural abstractions present at language-side.
	\Eye allows us to rapidly define customized views for \PH objects.
	Our inspector framework seamlessly integrates into the existing \PH debugger.
		
\end{description}


% =============================================================================
\section{Outline}
% =============================================================================

\begin{description}
\item[\chapref{background}] sheds light on the context of this work.
	We present a quick overview of language-side reflection followed by a development of \VM-level reflection.
	We find that mostly metacircular \VMs provide limited \VM-level reflection and thus we present several high-level language \VMs falling into this category.
	We conclude that there is only two research \VM that has a uniform model for \VM and language-side.
	Among them is \P a research \ST \VM we contributed to previous to working on this dissertation.

\item[\chapref{reification}] focuses on language-side applications that simplify the interaction with the underlying \VM.
	We present a custom inspector framework that is now used by default in \PH.
	As a second part we explain how we introduced first-class layouts and slots to \PH to reify the low-level structural layout of objects.
	Both projects are crucial for metacircular \VM development and are direct results from the research conducted on the \P \VM.

\item[\chapref{benzo}] describes a high-level low-level programming framework named \B.
	The core functionality of \B is to dynamically execute native-code generated at language-side.
	\B allows us to hoist typical \VM plugins to the language-side.
	Furthermore we show how code caching makes \B efficient and users essentially only pay a one-time overhead for generating the native code.
	
\item[\chapref{ffi}] presents \NB, a stable foreign function interface (\FFI) implementation that is entirely written at language-side using \B.
	\NB is a real-world validation of \B as it combines both language-side flexibility with \VM-level performance.
	We show in detail how \NB outperforms other existing \FFI solutions on \PH.

\item[\chapref{validation}] focuses on two further \B applications.
	In the first part we present \WF a framework for dynamically generating primitives at runtime.
	\WF extends the concept of metacircularity to the running language by reusing the same sources for dynamic primitives that were previously used to generate the static \VM artifact.
	In a first validation we show how \WF outperforms other reflective language-side solutions to instrument primitives.
	
	In a second part of \chapref{validation} we present \NBJ a prototype \JIT compiler that is based on \B.
	\NBJ shows the limitations of the \B approach as it required a customized \VM to communicate with the existing \JIT interface for native code.
	Our prototype implementation generates the same native code as the existing \VM-level \JIT, however, it is currently limited to simple expressions.
	\NBJ shows that for certain applications a well-define interface with the low-level components of the \VM is required.

\item[\chapref{future}] summarizes the limitations of \B and its application, furthermore we list undergoing efforts on the \B infrastructure and future work.

\item[\chapref{conclusion}] concludes the dissertation.

\end{description}

% =============================================================================
\ifx\wholebook\relax\else
    \end{document}
\fi
% =============================================================================