\documentclass[a4paper,12pt,twoside]{../includes/ThesisStyle}
\usepackage[utf8]{inputenc}
\usepackage[T1]{fontenc}

\usepackage[left=1.5in,right=1.3in,top=1.1in,bottom=1.1in,includefoot,includehead,headheight=13.6pt]{geometry}\renewcommand{\baselinestretch}{1.05}


% =============================================================================
%\usepackage[sectionbib]{chapterbib}	% Cross-reference package (Natural BiB)
%\usepackage{bibunits}
%\usepackage{natbib}					% Put References at the end of each chapter
\usepackage{algorithm}
\usepackage{alltt}
\usepackage{amsfonts}
\usepackage{amsmath}
\usepackage{amssymb}
\usepackage{cite}
\usepackage{color}
\usepackage{enumerate}
\usepackage{booktabs} % used for \midrule
\usepackage{fancyhdr}					% Fancy Header and Footer
\usepackage{graphicx}
\usepackage{ifthen}
\usepackage{latexsym}
\usepackage{multirow}
\usepackage{rotating}					% Sideways of figures & tables
\usepackage{stmaryrd}
\usepackage{subfigure}
\usepackage{url}         
\usepackage{xspace}
\usepackage[normalem]{ulem} % for \sout
\usepackage{xcolor}
\usepackage{tablefootnote}
\usepackage{pifont}

% =============================================================================

% Table of contents for each chapter
\usepackage[nottoc, notlof, notlot]{tocbibind}
\usepackage{minitoc}
\setcounter{minitocdepth}{1}
\mtcindent=15pt

\setcounter{secnumdepth}{3}
\setcounter{tocdepth}{2}
  
% =============================================================================
% Fancy Header Style Options

\pagestyle{fancy}                       % Sets fancy header and footer
\fancyfoot{}                            % Delete current footer settings

%\renewcommand{\chaptermark}[1]{         % Lower Case Chapter marker style
%  \markboth{\chaptername\ \thechapter.\ #1}}{}} %

%\renewcommand{\sectionmark}[1]{         % Lower case Section marker style
%  \markright{\thesection.\ #1}}         %

\fancyhead[LE,RO]{\bfseries\thepage}    % Page number (boldface) in left on even
% pages and right on odd pages
\fancyhead[RE]{\bfseries\nouppercase{\leftmark}}      % Chapter in the right on even pages
\fancyhead[LO]{\bfseries\nouppercase{\rightmark}}     % Section in the left on odd pages

\let\headruleORIG\headrule
\renewcommand{\headrule}{\color{black} \headruleORIG}
\renewcommand{\headrulewidth}{1.0pt}
\usepackage{colortbl}
\arrayrulecolor{black}

\fancypagestyle{plain}{
  \fancyhead{}
  \fancyfoot{}
  \renewcommand{\headrulewidth}{0pt}
}


% =============================================================================
% Clear Header Style on the Last Empty Odd pages
\makeatletter

\def\cleardoublepage{\clearpage\if@twoside \ifodd\c@page\else%
  \hbox{}%
  \thispagestyle{empty}%              % Empty header styles
  \newpage%
  \if@twocolumn\hbox{}\newpage\fi\fi\fi}

\makeatother

\newenvironment{maxime}[1]
{
\vspace*{0cm}
\hfill
\begin{minipage}{0.5\textwidth}%
%\rule[0.5ex]{\textwidth}{0.1mm}\\%
\hrulefill $\:$ {\bf #1}\\
%\vspace*{-0.25cm}
\it 
}%
{%

\hrulefill
\vspace*{0.5cm}%
\end{minipage}
}

\let\minitocORIG\minitoc
\renewcommand{\minitoc}{\minitocORIG \vspace{1.5em}}


\renewcommand{\epsilon}{\varepsilon}

% centered page environment
\newenvironment{vcenterpage}
	{\newpage\vspace*{\fill}\thispagestyle{empty}\renewcommand{\headrulewidth}{0pt}}
	{\vspace*{\fill}}
	

%=============================================================================

\usepackage{needspace}
\newcommand{\needlines}[1]{\Needspace{#1\baselineskip}}

\usepackage{xcolor}
\definecolor{source}{gray}{0.95}
% source code formatting
\usepackage{listings}
    % global settings for source code listing package
\lstset{
    basicstyle=\ttfamily\small,
    showspaces=false,
    showstringspaces=false,
    captionpos=b, 
    columns=fullflexible}

\lstdefinelanguage{ST}{
    keywordsprefix=\#,
    morekeywords=[0]{true,false,nil},
    morekeywords=[1]{self,super,thisContext},
    morekeywords=[2]{ifTrue:,ifFalse:,whileTrue:,whileFalse:,and:,or:,xor:,not:,by:,timesRepeat:},
    sensitive=true,
    morecomment=[s]{"}{"},
    morestring=[d]',
    escapechar={!},
    alsoletter={., :, -, =, +, <},
    moredelim=**[is][\itshape]{/+}{+/},
    literate=
        {^}{{$\uparrow$}}1
        {:=}{{$\leftarrow$}}1
        {~}{{$\sim$}}1
        {-}{{\sf -\hspace{-0.13em}-}}1  % the goal is to make - the same width as +
        {+}{\raisebox{0.08ex}{+}}1		% and to raise + off the baseline to match V
        , % Don't forget the comma at the end!
    style=STStyle
}
\lstdefinestyle{STStyle}{
    tabsize=4,
    %frame=leftline,
    % frame=bl,
    %framerule=2pt,
    %rulecolor=\color{gray},
    % backgroundcolor=\color{white},
    %backgroundcolor=\usebeamercolor[bg]{listing},
    basicstyle=\ttfamily\small,
    keywordstyle=\bf\ttfamily,
    % stringstyle=\color{orange},
    stringstyle=\mdseries\slshape,
    commentstyle=\it\rmfamily\color{darkgray}, 
    commentstyle=\mdseries\slshape\color{gray},
    %commentstyle=\mdseries\slshape,
    emphstyle=\bf\ttfamily,
    escapeinside={!}{!},
	%backgroundcolor=\color{source},
    %emphstyle={[2]\color{red}},
    %emphstyle={[3]\color{blue}\bf},
    %emphstyle={[4]\color{blue}},
    keepspaces=true
} 

%\lstnewenvironment{javacode}  [1][]{\lstset{language=java,#1}\needlines{#2}}{} 
%\lstnewenvironment{pythoncode}[2][]{\lstset{language=python,#1}\needlines{#2}}{}
\lstnewenvironment{stcode}    [2][]{\lstset{language=ST,#1}\needlines{#2}}{}
\lstnewenvironment{ccode}     [2][]
    {\lstset{language=C,numbers=left,escapechar=\$,numberstyle=\tiny,#1}\needlines{#2}}{}

% ON: I tried to pass the line number options in as arg #1 but it does not work for me
% I also could net get the line numbers to consistently increase
\lstnewenvironment{numstcode} [2][]
    {\lstset{language=ST,numbers=left,numberstyle=\tiny,numbersep=2pt,#1}\needlines{#2}}{}
\lstnewenvironment{numstcodecont} [2][]
    {\lstset{language=ST,numbers=left,numberstyle=\tiny,numbersep=2pt,firstnumber=last#1}\needlines{#2}}{}

\newcommand{\lst}[1]{{\tt #1}}

% In-line code (literal)

% In-line code (latex enabled)
% Use this only in special situations where \ct does not work
% (within Section headings ...):
\newcommand{\lct}[1]{{\textsf{\textup{#1}}}}
% Code environments
\lstnewenvironment{code}{%
	\lstset{%
		% frame=lines,
		frame=single,
		framerule=0pt,
		mathescape=false
	}
}{}

%\renewcommand{\lstlistingname}{Code Example}

% =============================================================================
\newboolean{showcomments}
\setboolean{showcomments}{true}

\ifthenelse{\boolean{showcomments}} {
	\newcommand{\ugh}[1] {\textcolor{red}{\uwave{#1}}}	% please rephrase
	\newcommand{\ins}[1] {\textcolor{blue}{\uline{#1}}}	% please insert
	\newcommand{\del}[1] {\textcolor{red}{\sout{#1}}}	% please delete
	\newcommand{\chg}[2] {								% please change
		\textcolor{red}{\sout{#1}}{\ra}
		\textcolor{blue}{\uline{#2}}}
	\newcommand{\nbc}[3]{								% comment
		{\colorbox{#3}{\bfseries\sffamily\scriptsize\textcolor{white}{#1}}}
		{\textcolor{#3}{\sf\small$\blacktriangleright$\textit{#2}$\blacktriangleleft$}}}

}{
	\newcommand{\ugh}[1]{#1}							% please rephrase
	\newcommand{\ins}[1]{#1}							% please insert
	\newcommand{\del}[1]{}								% please delete
	\newcommand{\chg}[2]{#2}							% please change
	\newcommand{\nbc}[3]{}								% comment
}

% =============================================================================
\usepackage[pagebackref,hyperindex=true]{hyperref}


% Links in pdf
\usepackage{color}
\definecolor{linkcol}{rgb}{0.0, 0.0, 0.0} 
\definecolor{citecol}{rgb}{0.0, 0.0, 0.0} 

% Change this to change the informations included in the pdf file
% See hyperref documentation for information on those parameters
\hypersetup {
	bookmarksopen=true,
	pdftitle="Design and Use of Anatomical Atlases for Radiotherapy",
	pdfauthor="Olivier COMMOWICK", 
	pdfsubject="Creation of atlases and atlas based segmentation", %subject of the document
	%pdftoolbar=false, % toolbar hidden
	pdfmenubar=true, %menubar shown
	pdfhighlight=/O, %effect of clicking on a link
	colorlinks=true,
	pdfpagemode=UseNone,
	pdfpagelayout=SinglePage,
	pdffitwindow=true,
	linkcolor=linkcol,
	citecolor=citecol,
	urlcolor=linkcol
}

% =============================================================================
\newcommand{\figlabel}[1] {\label{fig:#1}}
\newcommand{\chaplabel}[1]{\label{chap:#1}}
\newcommand{\seclabel}[1] {\label{sec:#1}}
\newcommand{\tablabel}[1] {\label{tab:#1}}
\newcommand{\lstlabel}[1] {\label{lst:#1}}

\newcommand{\figref}[1] {Figure~\ref{fig:#1}}
\newcommand{\chapref}[1]{Chapter~\ref{sec:#1}}
\newcommand{\secref}[1] {Section~\ref{sec:#1}}
\newcommand{\tabref}[1] {Table~\ref{tab:#1}}
\newcommand{\lstref}[1] {Listing~\ref{tab:#1}}

\newcommand{\commented}[1]{}

\newcommand{\bs}    {\symbol{'134}} % backslash
\newcommand{\us}    {\symbol{'137}} % underscore
\newcommand{\ttt}[1]{\texttt{#1}}
\newcommand{\ie}    {\emph{i.e.},\xspace}
\newcommand{\eg}    {\emph{e.g.},\xspace}
\newcommand{\etal}  {\emph{et al.}\xspace}
\newcommand{\ns}    {\!\!\!\!} %big negative space
\newcommand{\cnull} {\textbackslash0\xspace}


\newcommand\fix[1]{\nb{FIX}{#1}}
\newcommand\todo[1]{\nb{TO DO}{#1}}
\newcommand\cb[1]{\nbc{CB}{#1}{purple}}
\newcommand\sd[1]{\nbc{SD}{#1}{orange}}
\newcommand\is[1]{\nbc{IS}{#1}{gray}}
\newcommand\gc[1]{\nbc{GC}{#1}{olive}}
\newcommand\ct[1]{\nbc{CT}{#1}{teal}}
\newcommand\md[1]{\nbc{MD}{#1}{blue}}
\newcommand\dc[1]{\nbc{DC}{#1}{green}}

% =============================================================================
\newcommand{\NBFFI}  {Native\-Boost-FFI\xspace}
\newcommand{\NB}  {Native\-Boost\xspace}
\newcommand{\B}   {Benzo\xspace}
\newcommand{\ST}  {Small\-talk\xspace}
\newcommand{\PH}  {Pharo\xspace}
\graphicspath{{.}{../figures/}}

\begin{document}
% =============================================================================
\chapter{Appendix}
\markboth{Appendix}{Appendix}
\chaplabel{appendix}
%\minitoc
% =============================================================================

%\clearpage

\vspace*{5cm}
% =============================================================================
\section{\PH Programming Language}
\seclabel{pharo}
% =============================================================================


\PH is a \ST inspired object-oriented and dynamically-typed general-purpose language with its own programming environment.
The language has a simple and expressive syntax which can be learned in a few minutes.
Concepts in \PH are very consistent, everything is an object: classes, methods, numbers, strings, even the execution context.

\PH runs on top of a bytecode-based \emph{virtual machine}.
Development takes place in an \emph{image} in which all objects reside.
All these objects can be modified by the programmer, this includes classes and methods.
Hence, we eliminate the typical edit/compile/run cycle and instead incrementally add, remove or modify classes and methods.
It is worth noting that \emph{all} classes can be extended with new methods in \PH.
For instance, one can add new operations on integers or strings, classes that are treated as unchangeable internal objects by many other high-level languages.
For deployment and debugging, the state of a running image can be saved at any point and subsequently restored.

\newpage

% ---------------------------------------------------------------------------
\subsection{Minimal Syntax}
% ---------------------------------------------------------------------------

\noindent
\begin{tabularx}{\linewidth}{@{}rX@{}}
	\multicolumn{2}{l}{Reserved Words}\\
	\midrule
	\textcolor{darkRed}{\texttt{nil}} & the undefined object\\
	\textcolor{darkRed}{\texttt{true}}, \textcolor{darkRed}{\texttt{false}} & boolean objects\\
	\textcolor{darkCyan}{\texttt{self}} & the receiver of the current message\\
	\textcolor{darkCyan}{\texttt{super}} & the receiver, in the superclass context\\
	\textcolor{darkCyan}{\texttt{thisContext}} & the current invocation on the call stack \\
	\\
	\multicolumn{2}{l}{Literal Object Syntax}\\
	\midrule
	\textcolor{string}{\texttt{'}{a string}\texttt{'}} & \\
	\textcolor{string}{\texttt{\#symbol}} & unique string \\
	\textcolor{darkRed}{\texttt{\$a}} & the character \textcolor{darkRed}{\texttt{a}} \\
	\textcolor{darkRed}{\texttt{12 2r1100 16rC}} & integers twelve in decimal, binary and hexadecimal encoding\\
	\textcolor{darkRed}{\texttt{3.14 1.2e3}} & floating-point numbers\\
	\texttt{\#(\textcolor{string}{abc} \textcolor{darkRed}{123})} & literal array containing the symbol \textcolor{string}{\texttt{\#abc}} and the number \textcolor{darkRed}{\texttt{123}} \\
	\texttt{\#[\textcolor{darkRed}{12} \textcolor{darkRed}{16rFF}]} & literal byte array containing the bytes/integers \textcolor{darkRed}{12} and \textcolor{darkRed}{255}\\
	\texttt{\{\textcolor{darkBlue}{foo}\,.\ \textcolor{darkRed}{3}\,+\,\textcolor{darkRed}{2}\}} & dynamic array built from 2 expressions\\
	
	\\
	\multicolumn{2}{l}{Reserved Characters in Expressions}\\
	\midrule
	\textcolor{comment}{\texttt{"}{a comment}\texttt{"}} & \\
	\texttt{.} & expression separator (period)\\
	\texttt{;} & message cascade (semicolon)\\
	\texttt{:=} & {assignment} \\
	\texttt{\textasciicircum} & return a result from a method (caret)\\
	\texttt{[\,:\textcolor{darkBlue}{p}\,|\,}\emph{expr}\texttt{\,]} & code block with a parameter \\
	\texttt{|\,\textcolor{darkBlue}{foo bar}\,|} & declaration of two temporary variables\\
	\texttt{<pragma>}, \texttt{<primitive: 3>} & pragma or annotations used in methods, for instances to declare a primitive method.
\end{tabularx}

% ---------------------------------------------------------------------------
\subsection{Message Sending}
% ---------------------------------------------------------------------------

A method is called by sending a message to an object called the \emph{receiver}.
Each message returns an object.
Messages are modeled from natural languages with a subject a verb and complements. There are three types of messages with descending precedence: unary, binary, and keyword.

\begin{description}
\item[Unary messages] have no arguments.

\begin{alltt}
\textcolor{darkBlue}{Array} new.
\end{alltt}

The first example creates and returns a new instance of the \textcolor{darkBlue}{\texttt{Array}} class, by sending the message \texttt{new} to the class
\textcolor{darkBlue}{\texttt{Array}} that is an object.

\begin{alltt}
#(\textcolor{darkRed}{1 2 3}) size.
\end{alltt}

The second message returns the size of the literal array which is \textcolor{darkRed}{\texttt{3}}.

\item[Binary messages] take only one argument and are named by one or more symbol characters.

\begin{alltt}
\textcolor{darkRed}{3} + \textcolor{darkRed}{4}.
\end{alltt}

The \texttt{+} message is sent to the integer object \textcolor{darkRed}{\texttt{3}} with \textcolor{darkRed}{\texttt{4}} as the argument.

\begin{alltt}
\textcolor{string}{'Hello'}, \textcolor{string}{' World'}.
\end{alltt}

In the second case, the string \textcolor{string}{\texttt{'Hello'}} receives the message \texttt{,} (comma) with the string \textcolor{string}{\texttt{'~World'}} as the argument.

\item[Keyword messages] can take one or more arguments that are inserted in the message name.

\begin{alltt}
\textcolor{string}{'Smalltalk'} allButFirst: \textcolor{darkRed}{5}.
\end{alltt}

The first example sends the message \texttt{allButFirst:} to a string, with the argument \textcolor{darkRed}{\texttt{5}}.
This returns the string \textcolor{string}{\texttt{'talk'}}.

\begin{alltt}
\textcolor{darkRed}{3} to: \textcolor{darkRed}{10} by: \textcolor{darkRed}{2}.
\end{alltt}

The second example sends \texttt{to:by:} to \textcolor{darkRed}{\texttt{3}}, with arguments \textcolor{darkRed}{\texttt{10}} and \textcolor{darkRed}{\texttt{2}}; this returns a collection containing \textcolor{darkRed}{\texttt{3}}, \textcolor{darkRed}{\texttt{5}}, \textcolor{darkRed}{\texttt{7}}, and \textcolor{darkRed}{\texttt{9}}.

\end{description}


% ---------------------------------------------------------------------------
\subsection{Precedence}
% ---------------------------------------------------------------------------

There is a fixed global precedence when evaluating expressions in \PH: Parentheses\,$>$\,unary\,$>$\,binary\,$>$\,keyword, and finally from left to right.

\begin{alltt}
(\textcolor{darkRed}{10} between: \textcolor{darkRed}{1} and: \textcolor{darkRed}{2}\,+\,\textcolor{darkRed}{4}\,*\,\textcolor{darkRed}{3}) not
\end{alltt}

Here, the messages \texttt{+} and \texttt{*} are sent first, then \texttt{between:and:} is sent, and finally \texttt{not}.
The rule suffers no exception: operators are just binary messages with \emph{no notion of mathematical precedence}, so \texttt{\textcolor{darkRed}{2}\,+\,\textcolor{darkRed}{4}\,*\,\textcolor{darkRed}{3}} reads left-to-right and thus yields \textcolor{darkRed}{18} and not the expected \textcolor{darkRed}{14}!

% ---------------------------------------------------------------------------
\subsection{Cascading Messages}
% ---------------------------------------------------------------------------

Multiple messages can be sent to the same receiver with \texttt{;}.

\begin{alltt}
\textcolor{darkBlue}{OrderedCollection} new
  add: \textcolor{string}{#abc};
  add: \textcolor{string}{#def};
  add: \textcolor{string}{#ghi}.
\end{alltt}

The message \texttt{new} is sent to \texttt{\textcolor{darkBlue}{OrderedCollection}} which
results in a new collection to which three \texttt{add:} messages are sent with different arguments.
The value of the whole message cascade is the value of the last message sent (here, the symbol \textcolor{string}{\texttt{\#ghi}}).
This example is the equivalent of first assigning the new collection to a temporary variable and sending three separate \texttt{add:} messages:

\begin{alltt}
| newCollection | 
newCollection := \textcolor{darkBlue}{OrderedCollection} new.
newCollection add: \textcolor{string}{#abc}.
newCollection add: \textcolor{string}{#def}.
newCollection add: \textcolor{string}{#ghi}.
\end{alltt}


To return the original receiver of the message cascade (\ie the collection) instead of the last result (\ie \textcolor{string}{\texttt{\#ghi}}), the \texttt{yourself} message is used:

\begin{alltt}
\textcolor{darkBlue}{OrderedCollection} new
  add: \textcolor{string}{#abc};
  add: \textcolor{string}{#def};
  add: \textcolor{string}{#ghi};
  yourself.
\end{alltt}

% ---------------------------------------------------------------------------
\subsection{Blocks}
% ---------------------------------------------------------------------------

Blocks are objects containing code that is executed on demand,
(anonymous functions or closures).
They are the basis for control structures like conditionals and loops.

\begin{alltt}
\textcolor{darkRed}{2} = \textcolor{darkRed}{2}
  ifTrue: [ \textcolor{darkBlue}{Error} signal: \textcolor{string}{'Help'} ].
\end{alltt}

The first example sends the message \texttt{ifTrue:} to the boolean
\textcolor{darkRed}{\texttt{true}} (computed from \texttt{\textcolor{darkRed}{2} = \textcolor{darkRed}{2}}) with a block as argument.
Because the boolean is \textcolor{darkRed}{\texttt{true}}, the block is executed and an exception is signaled.

\begin{alltt}
\#(\textcolor{string}{'Hello World'} \textcolor{darkRed}{\$!})
  do: [ :\textcolor{darkBlue}{e} | \textcolor{darkBlue}{Transcript} show: \textcolor{darkBlue}{e} ]
\end{alltt}

The next example sends the message \texttt{do:} to an array.
This evaluates the block once for each element, passing it via the \texttt{e} parameter.
As a result, \texttt{\textcolor{string}{Hello~World!}} is printed.


% ---------------------------------------------------------------------------
\subsection{Methods}
% ---------------------------------------------------------------------------

Methods are first-class objects in \PH and can be inspected and modified on the fly.
Methods are created by saving expressions in the \PH development environment.
Typically methods are printed with a special first line indicating the class the method is installed on and the name or selector it is given.

\begin{alltt}
\textcolor{darkBlue}{Array} >> helpMethod
    \textcolor{darkRed}{2} = \textcolor{darkRed}{2}
        ifTrue: [ \textcolor{darkBlue}{Error} signal: \textcolor{string}{'Help'} ].
\end{alltt}

This example would denote a simple method with a unary selector on the \texttt{Array} class.
This method could be invoked by evaluating \texttt{\textcolor{darkBlue}{Array} new helpMethod}.

Certain methods are marked with a pragma to use predefined primitives from the \VM.
These are used for expressions that cannot be expressed in \PH.
For instance the \texttt{basicNew} which allocates new objects uses the primitive number 70:

\begin{alltt}
\textcolor{darkBlue}{Behavior} >> basicNew
    \textcolor{comment}{"Answer a new instance of this class"}
    <primitive: \textcolor{darkRed}{70}>
    \textcolor{darkBlue}{OutOfMemory} signal.
\end{alltt}


% =============================================================================
\ifx\wholebook\relax\else
    \end{document}
\fi