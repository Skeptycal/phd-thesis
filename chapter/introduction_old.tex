\noindent Reflection as common feature has been adopted by several dynamic high-level languages such as \urlfootnote{\Ruby}{http://rubylang.org/}, \urlfootnote{\Python}{http://python.org/} or \urlfootnote{\PH}{http://pharo.org/}.
Even some less dynamic languages such as \Java support a broad range of reflective features.
In the context of this dissertation we will mainly refer to the \PH environment which inherited its reflective capability from \ST.
One major reservation of programming languages over reflection is performance.
Reflection requires support from the underlying \VM and introduces another level of late-binding which prevent more aggressive static optimizations.

%\todo{probably drop this paragraph}
%Typically we distinguish two types of reflection: structural and behavioral \cite{Maes87a}.
%Structural reflection is concerned with the static structure of a program, while behavioral reflection focuses on the dynamic part of a running program.
%Orthogonally to the previous categorization we distinguish between introspection and intercession. 
%For introspection we only access a reified concept, whereas for intercession we alter the reified representation.

Especially behavioral reflection is costly, as it does not access the statically defined structure but requires dynamically reified representations of the execution context of a program.
There several approaches to limit the costs of reflection and the impact on the whole language runtime:
\begin{itemize}[nolistsep]
	\item Limit the amount of reification performed,
	\item Restrict the reflective capabilities,
	\item Optimize the \VM to lower the reification overhead.
\end{itemize}
The latter case allows more optimizations and is typically the path chose for more performance oriented programming languages.
For \PH only the first option is feasible as the integrated development environment makes ready use of reflection.
The first approach results in \emph{partial reflection} or \emph{partial behavior reflection} \cite{Tant03a}.
Typically the code locations where reification is necessary do not change often and can thus by dynamically optimized at runtime \cite{Roet07b}.

The research on partial behavioral reflection shows the real world limitations of reflective applications.
For many applications the reification overhead is too big.
The third approach to limit the costs is by providing specific support for a required reification.
An example for that is the debugging capability of many programming languages.
Most programming have special \VM support for this.
In contrast, \PH or \ST systems the debugger is built on top of the default reflection infrastructure.
\todo{are there other similar possiblities}

In general we require \VM-level support for efficient reflection.
Intercession is contained at language-side and there is usually no interface present to reach the \VM.
Hence, custom support is only possible by statically modifying the underlying \VM upfront.
Even in a theoretical setup where the \VM is open for language-side modification we see that hardly any \VM is self-aware.
Most \VMs behave at runtime like any other binary and lack most structural information required for reflection.
A partial exception to this are \VMs written on top of \VM frameworks that use high-level languages to describe the \VM components instead of the prevailing static system programming languages such as C or C++.
Even though, such \VM frameworks allow for compile-time reflection, the reflective capabilities are lost during compilation.

All these \VMs have in common that they strongly isolate the language running on top.
Recently several language-runtimes appeared that no longer make this strong separation \cite{Unga05a, Verw12a}.
One such system is the \P \ST \VM.
Instead of using bytecodes as language-side execution format \P's compiler directly generates native code for execution.
Essentially \P hoist the \VM-level just-in-time compiler (\JIT) to the language-side.
At the same time the underlying \VM is reduced to a bootstrap routine that hands over execution to the \P generated native code.
The direct usage of native code allows us to specify the lookup routine in plain \ST code at language-side.

Another project that follows the same principles as \P is the \Klein \VM.
Both language-runtimes have a reified base-level, the \VM components are represented as language-side structures.
This allows us to modify and inspect the \VM, it is reflective.
Both \Klein and \P are whole-system bottom up approaches that add powerful hooks at language-side.
Is it possible to achieve similar functionality with a simpler interface?
What are the limiting factors for such an approach?

To answer these questions we propose the high-level low-level programming framework \B written for \PH.
In the core \B allows us to dynamically activate native code from language-side.
This adds a primitive yet generic interface to the low-level \VM world.
To validate \B we describe three distinct applications built on top of it.
%
\begin{description}
	\item[\FFI:] The first one is an efficient foreign function interface (\FFI) library that is built at language-side without additional \VM support.
	Our \FFI library outperforms existing solutions on \PH.
	
	\item[Dynamic Primitives:] The second applications uses \B to dynamically generate and modify \PH primitives by reusing the metacircular \VM sources.
	By combining high-level reflection and \B's low-level performance we outperform pure \PH-based primitive instrumentation.
	
	\item[Language-side \JIT:] As a third, prototype application we show how \B is used to build a language-side \JIT.
	Our prototype shows the limits of possible \VM interactions using the \B framework. 
\end{description}